%{Лекція 2}

\subsubsection{Метод послідовних наближень для інтегральних рівнянь з полярним ядром}

Ядро $K(x, y)$ називається полярним, якщо воно представляється у вигляді:
\begin{equation}
	\label{eq:1.20}
	K(x, y) = \dfrac{A(x, y)}{|x - y|^\alpha}
\end{equation}
де $A \in C(\bar G \times \bar G)$, $|x - y| = \left( \sum_{i = 1}^n (x_i - y_i)^2 \right)^{1/2}$, $\alpha < n$ ($n$ -- розмірність евклідового простору). \\

Ядро називається слабо полярним, якщо $\alpha < n / 2$. \\

Метод послідовних наближень для інтегральних рівнянь з неперервним ядром мав вигляд: 
\begin{multline*}
\phi(x) = \lambda \Int_G K(x, y) \phi(x, y) \diff y + f(x), \\
\phi_0 = f, \quad \phi_1 = f + \lambda \bf{K} \phi_0, \quad \ldots, \quad \phi_{n + 1} = f + \lambda \bf{K} \phi_n.
\end{multline*}

Оцінки, що застосовувались для неперервних ядер не працюють для полярних ядер, тому що максимум полярного ядра рівний нескінченності (ядро необмежене в рівномірній метриці), отже, сформулюємо лему аналогічну лемі 1 для полярних ядер. 
\begin{lemma}
	Інтегральний оператор $\bf{K}$ з полярним ядром $K(x, y)$ переводить множину функцій $C(\bar G) \xrightarrow{\bf{K}} C (\bar G)$ і при цьому має місце оцінка: 
	\begin{equation}
		\label{eq:1.21}
		\| \bf{K} \phi \|_{C(\bar G)} \le N \| \phi\|_{C(\bar G)},
	\end{equation}
	де 
	\begin{equation}
		\label{eq:1.22}
		N = \Max_{x \in \bar G} \Int_G |K(x, y) \diff y|.
	\end{equation}
\end{lemma}
\begin{proof}
	Спочатку доведемо, що функція $\bf{K}\phi$ неперервна в точці $x_0$. \\

	Оцінимо при умові $|x - x_0| < \eta / 2$ вираз:
	\begin{multline*}
		\left| \Int_G K(x, y) \phi(y) \diff y - \Int_G K(x_0, y) \phi(y) \diff y \right| = \\
		= \left| \Int_G \dfrac{A(x, y)}{|x - y|^\alpha} \phi(y) \diff y - \Int_G \dfrac{A(x_0, y)}{|x_0 - y|^\alpha} \phi(y) \diff y \right| \le \\
		\le \Int_G \left|\dfrac{A(x, y)}{|x - y|^\alpha} - \dfrac{A(x_0, y)}{|x_0 - y|^\alpha}\right| |\phi(y)| \diff y \le (*)
	\end{multline*}
	винесемо $\max \phi(y)$ у вигляді $\|\phi\|_{C(\bar G)}$, а інтеграл розіб’ємо на два інтеграли: інтеграл по $U(x_0, \eta)$ -- кулі з центром в $x_0$ і радіусом $\eta$; інтеграл по залишку $G \setminus U(x_0, \eta)$.
	
	\begin{multline*} 
		(*) \le \|\phi\|_{C(\bar G)} \left( \Int_{U(x_0, \eta)} \left|\dfrac{A(x, y)}{|x - y|^\alpha} - \dfrac{A(x_0, y)}{|x_0 - y|^\alpha}\right| \diff y\right. + \\
		+ \left.\Int_{G \setminus U(x_0, \eta)} \left|\dfrac{A(x, y)}{|x - y|^\alpha} - \dfrac{A(x_0, y)}{|x_0 - y|^\alpha}\right| \diff y\right)
	\end{multline*}
	
	Оцінимо тепер кожний з інтегралів:
	
	\[ \Int_{U(x_0, \eta)} \left|\dfrac{A(x, y)}{|x - y|^\alpha} - \dfrac{A(x_0, y)}{|x_0 - y|^\alpha}\right| \diff y \le A_0 \Int_{U(x_0, \eta)} \left|\dfrac{\diff y}{|x - y|^\alpha} - \dfrac{\diff y}{|x_0 - y|^\alpha}\right|, \]

	де $A_0$ -- $\max$ функції $A(x, y)$ на потрібній множині. \\ 

	Введемо узагальнені сферичні координати з центром у точці $x_0$ в просторі $\RR^n$:
	\begin{align*} 
		y_1 &= x_{0, 1} + \rho \cos \nu_1 \\
		y_2 &= x_{0, 2} + \rho \sin \nu_1 \cos \nu_2 \\
		\ldots \\
		y_{n - 1} &= x_{0, n - 1} + \rho \sin \nu_1 \cdot \ldots \cdot \cos \nu_{n - 1} \\
		y_n &= x_{0, n} + \rho \sin \nu_1 \cdot \ldots \cdot \sin \nu_{n - 1}
	\end{align*}

	Якобіан переходу має вигляд:
	\[ \dfrac{D(y_1, \ldots, y_n)}{\rho, \nu_1, \ldots, \nu_{n - 1}} = \rho^{n - 1} \Phi(\sin \nu_1, \ldots, \sin \nu_{n - 1}, \cos \nu_1, \ldots, \cos \nu_{n - 1}), \]

	де $0 \le \rho \le \eta, 0 \le \nu_i \le \pi, i = \overline{1, n - 2}, 0 \le \nu_{n - 1} \le 2 \pi$. \\

	Отримаємо \[ \Int_{U(x_0, \eta)} \dfrac{\diff y}{|x_0 - y|^\alpha} = \sigma_n \Int_0^\eta \dfrac{\rho^{n - 1} \diff \rho}{\rho^\alpha} = \sigma_n \left.\dfrac{\rho^{n - \alpha}}{n - \alpha}\right|_0^\eta = \dfrac{\sigma_n \eta^{n - \alpha}}{n - \alpha} \le \dfrac{\epsilon}{4}, \]
	де $\sigma_n$ -- площа поверхні одиничної сфери в $n$-вимірному просторі $\RR^n$. \\

	Оскільки $|x - x_0| < \eta / 2$, то \[ \Int_{U(x_0, \eta)} \dfrac{\diff y}{|x - y|^\alpha} \le \Int_{U(x_0, 3\eta/2)} \dfrac{\diff y}{|x_0 - y|^\alpha} \le \dfrac{\sigma_n}{n - \alpha} \left(\dfrac{3\eta}{2}\right)^{n - \alpha} \le \dfrac{\epsilon}{4}. \] 

	Оскільки $\frac{A(x, y)}{|x - y|^\alpha} \in C\left(\overline{U (x_0, \eta/2)}\times\overline{G \setminus U (x_0, \eta)}\right)$, то
	\[ \Int_{G \setminus U(x_0, \eta)} \left|\dfrac{A(x, y)}{|x - y|^\alpha} - \dfrac{A(x_0, y)}{|x_0 - y|^\alpha}\right| \diff y \le \dfrac{\epsilon}{2}. \]

	Таким чином ми довели, що $\left| \int_G K(x, y) \phi(y) \diff y - \int_G K(x_0, y) \phi(y) \diff y \right| \le \epsilon$, тобто функція $\bf{K}\phi$ неперервна в точці $x_0$. \\

	Доведемо оцінку $\| \bf{K}\phi \|_{C(\bar G)} \le N \|\phi\_{C(\bar G)}$, де $N = \max_{x \in \bar G} \int_G |K(x, y) \diff y|$:
	\begin{multline*}
		\left| \Int_G K(x, y) \phi(y) \diff y \right| \le \Int_G |K(x, y)| |\phi(y)| \diff y \le \|\phi\|_{C(\bar G)} \Int_G |K(x, y)| \le \\
		\le |\phi\|_{C(\bar G)} \Max_{x \in \bar G} \Int_G |K(x, y)| \diff y = N \|\phi\|_{C(\bar G)},
	\end{multline*}
	отже $\| \bf{K}\phi \|_{C(\bar G)} \le N \|\phi\|_{C(\bar G)}$. \\

	Покажемо скінченність $N = \max_{x \in \bar G} \int_G |K(x, y) \diff y|$. Розглянемо \[\Int_G |K(x,y)| \diff y \le A_0 \Int_G \dfrac{\diff y}{|x - y|^\alpha} \le (*).\]

	Для будь-якої точки $x$, існує радіус (рівний максимальному діаметру області $G$) такий, що в кулю з цим радіусом попадає будь-яка точка $y$: $D = \diam G$.

	\[ (*) \le A_0 \Int_{U(x, D)} \dfrac{\diff y}{|x - y|^\alpha} = A_0 \dfrac{\sigma_n}{n - \alpha} D^{n - \alpha}. \]
\end{proof}

\begin{theorem}[про існування розв’язку інтегрального рівняння Фредгольма з полярним ядром
для малих значень параметру]
	Інтегральне рівняння Фредгольма 2-го роду з полярним ядром $K(x, y)$ має єдиний розв’язок в класі неперервних функцій для будь-якого неперервного вільного члена $f$ при умові
	\begin{equation}
		\label{eq:1.23}
		|\lambda| < \dfrac{1}{N}
	\end{equation}
	і цей розв’язок може бути представлений рядом Неймана, який збігається абсолютно і рівномірно.
\end{theorem}
\begin{proof}
	Сформулюємо умову збіжності ряду Неймана. \\

	$\phi = \sum_{i = 0}^\infty \lambda^i \bf{K}^i f$, отже $\|\phi\|_{C(\bar G)} \le \sum_{i = 1}^\infty |\lambda|^i N^i \|f\|_{C(\bar G)}$. \\

	Останній ряд – геометрична прогресія і збігається при умові $|\lambda| < \frac{1}{N}$.
\end{proof}

\begin{lemma}
	Нехай маємо два полярних ядра $K_i(x, y) = \frac{A_i(x, y)}{|x - y|^\alpha_i}$, $\alpha_i < n$, $i = 1, 2$, а область $G$ обмежена, тоді ядро $K_3(x, y) = \int_G K_2(x, \xi) K_1(\xi, y) \diff \xi$ також полярне, причому має місце співвідношення:
	\begin{equation}
		\label{eq:1.24}
		K_3(x, y) = \begin{cases}
			\dfrac{A_3(x, y)}{|x - y|^{\alpha_1 + \alpha_2 - n}}, & \alpha_1 + \alpha_2 - n > 0, \\
			A_3(x, y) |\ln|x - y|| + B_3(x, y), & \alpha_1 + \alpha_2 - n = 0, \\
			A_3(x, y), & \alpha_1 + \alpha_2 - n < 0,
		\end{cases}
	\end{equation}
	де $A_3, B_3$ неперервні функції.
\end{lemma}

З леми 4 випливає, що всі повторні ядра $K_{(p)}(x, y)$, полярного ядра $K(x, y)$ задовольняють оцінкам: \\

$\alpha_1 = \alpha_2 = \alpha$.

\begin{equation}
	\label{eq:1.25}
	K_{(2)}(x, y) = \begin{cases}
		\dfrac{A_2(x, y)}{|x - y|^{2\alpha - n}}, & 2\alpha - n > 0, \\
		A_2(x, y) |\ln|x - y|| + B_2(x, y), & 2\alpha - n = 0, \\
		A_2(x, y), & 2\alpha - n < 0,
	\end{cases}
\end{equation}
\begin{equation}
	\label{eq:1.26}
	K_{(3)}(x, y) = \begin{cases}
		\dfrac{A_3(x, y)}{|x - y|^{3\alpha - 2n}}, & 3\alpha - 2n > 0, \\
		A_3(x, y) |\ln|x - y|| + B_3(x, y), & 3\alpha - 2n = 0, \\
		A_3(x, y), & 3\alpha - 2n < 0,
	\end{cases}
\end{equation}
\begin{equation}
	\label{eq:1.27}
	K_{(p)}(x, y) = \begin{cases}
		\dfrac{A_p(x, y)}{|x - y|^{p\alpha - (p-1)n}}, & p\alpha - (p - 1)n > 0, \\
		A_p(x, y) |\ln|x - y|| + B_p(x, y), & p\alpha - (p - 1)n = 0, \\
		A_p(x, y), & p\alpha - (p - 1)n < 0.
	\end{cases}
\end{equation}

Легко бачити, що для $\forall \alpha, n$ існує $p_0$ таке, що починаючи з нього всі повторні ядра є неперервні:
\begin{equation}
	\label{eq:1.28}
	p \alpha - (p - 1) n < 0 \Rightarrow (n - \alpha) p > n \Rightarrow p > \dfrac{n}{n - \alpha} \Rightarrow p_0 = \left[ \dfrac{n}{n - \alpha} \right] + 1.
\end{equation}

Звідси маємо, що резольвента $\mathcal{R}(x, y, \lambda)$ полярного ядра $K(x, y)$ складається з двох частин полярної складової $\mathcal{R}_1(x, y, \lambda)$ і неперервної складової $\mathcal{R}_2(x, y, \lambda)$:
\begin{multline}
	\label{eq:1.29}
	\mathcal{R}(x, y, \lambda) = \mathcal{R}_1(x, y, \lambda) + \mathcal{R}_2(x, y, \lambda) = \\
	= \Sum_{i = 1}^\infty \lambda^{i - 1} K_{(i)}(x, y) = \Sum_{i = 1}^{p_0 - 1} \lambda^{i - 1} K_{(i)}(x, y) + \Sum_{i = p_0}^\infty \lambda^{i - 1} K_{(i)}(x, y).
\end{multline}

Для доведення збіжності резольвенти, потрібно дослідити збіжність нескінченного ряду $\mathcal{R}_2(x, y, \lambda)$. Він сходиться рівномірно при $x, y \in \bar G$, $|\lambda| \le \frac{1}{N} - \epsilon$, $\forall \epsilon > 0$, визначаючи неперервну функцію $\mathcal{R}$ при $x, y \in \bar G$, $|\lambda| < \frac{1}{N}$ і аналітичну по $\lambda$ в крузі 
\begin{equation}
	\label{eq:1.30}
	|\lambda| < \dfrac{1}{N}.
\end{equation}

Дійсно \[\mathcal{R}_2(x, y, \lambda) = \Sum_{i = p_0}^\infty \lambda^{i - 1} K_{(i)}(x, y).\]

У свою чергу, \[|\lambda^{p_0 + s - 1} K_{(p_0 + s)}(x, y)| \le |\lambda|^{p_0 + s - 1} M_{p_0} N^s,\] де $M_{p_0} = \max_{x, y \in \bar G \times \bar G} |K_{p_0}(x, y)|$. Таким чином ряд $\mathcal{R}_2(x, y, \lambda)$ мажорується геометричною прогресією, яка збігається при умові (\ref{eq:1.30}).

\subsection{Теореми Фредгольма}
\subsubsection{Інтегральні рівняння з виродженим ядром}

\begin{definition*}
	Неперервне ядро $K(x, y)$ називається виродженим, якщо представляється у вигляді
	\begin{equation}
		\label{eq:2.1}
		K(x, y) = \Sum_{i = 1}^N f_i(x) g_i(y),
	\end{equation}
	де $\{ f_i \}_{i = \overline{1, N}}, \{ g_i \}_{i = \overline{1, N}} \subset C(\bar G)$, і $\{ f_i \}_{i = \overline{1, N}}$ та $\{ g_i \}_{i = \overline{1, N}}$, -- лінійно незалежні системи функцій.
\end{definition*}

Розглянемо інтегральні рівняння Фредгольма з виродженим ядром 
\begin{equation}
	\label{eq:2.2}
	\phi(x) = \lambda \Int_G K(x, u) \phi(y) \diff y + f(x).
\end{equation}

Підставимо вигляд ядра з (\ref{eq:2.1}) отримаємо:
\begin{multline}
	\label{eq:2.3}
	\phi(x) = \lambda \Int_G \Sum_{i = 1}^N f_i(x) g_i(y) \phi(y) \diff y + f(x) = \\
	= \lambda \Sum_{i = 1}^N f_i(x) \Int_G g_i(y) \phi(y) \diff y + f(x) = f(x) + \lambda \Sum_{i = 1}^N c_i f_i(x),
\end{multline}\
де 
\begin{equation}
	\label{eq:2.4}
	c_j = \Int_G g_j(y) \phi(y) \diff y.
\end{equation}
В (\ref{eq:2.4}) підставимо значення $\phi(x)$ з (\ref{eq:2.3}):

\begin{multline*}
	c_j = \Int_G g_j(y) \phi(y) \diff y = \Int_G g_j(y) \left( f(y) + \lambda \Sum_{i = 1}^N c_i f_i(y) \right) \diff y = \\
	= \Int_G g_j(y) f(y) \diff y + \lambda \Sum_{i = 1}^N c_i \Int_G g_j(y) f_i(y) \diff y.
\end{multline*}
В результаті отримаємо систему лінійних алгебраїчних рівнянь
\begin{equation}
	\label{eq:2.5}
	c_j = \lambda \Sum_{i = 1}^N \alpha_{j i} c_i + a_j, \quad j = \overline{1, N},
\end{equation}
де 
\begin{equation}
	\label{eq:2.6}
	\alpha_{ji} = \Int_G g_j(y) f_i(y) \diff y, \quad a_j = \Int_G g_j(y) f(y) \diff y.
\end{equation}
Отримаємо систему рівнянь для спряженого ядра:
\begin{equation}
	\label{eq:2.1'}
	K^*(x, y) = \Sum_{i = 1}^N \bar f_i(y) \bar g_i(x),
\end{equation}
\begin{equation}
	\label{eq:2.2'}
	\psi(x) = \bar \lambda \Int_G K^*(x, y) \psi(y) \diff y + g(x),
\end{equation}
\begin{equation}
	\label{eq:2.3'}
	\psi(x) = \bar \lambda \Sum_{i = 1}^N \bar g_i(x) \Int_G \bar f_i(y) \psi(y) \diff y + g(x) = \bar \lambda \Sum_{i = 1}^N d_i \bar g_i(x) + g(x),
\end{equation}
\begin{equation}
	\label{eq:2.4'}
	d_i = \Int_G \bar f_i(y) \psi(y) \diff y, \quad d_j = \Int_G \bar f_j(y) \left( g(y) + \bar \lambda \Sum_{i = 1}^N d_i \bar g_i(y) \right) \diff y,
\end{equation}
\begin{equation}
	\label{eq:2.5'}
	d_j = \bar \lambda \Sum_{i = 1}^N \beta_{ji}d_i + b_j, \quad i = \overline{1, N}, 
\end{equation}
\begin{equation}
	\label{eq:2.6'}
	\beta_{ji} = \Int_G \bar f_j(y) \bar g_i(y) \diff y, \quad b_j = \Int_G \bar f_j(y) g(y) \diff y,
\end{equation}
\begin{equation}
	\label{eq:2.7}
	\beta_{ji} = \bar \alpha_{ij}.
\end{equation}
Тобто отримуємо системи лінійних рівнянь які в матричному вигляді запишуться так:
\begin{equation}
	\label{eq:2.8}
	\vec c = \lambda A \vec c + \vec a,
\end{equation}
\begin{equation}
	\label{eq:2.8'}
	\vec d = \lambda A^* \vec d + \vec b,
\end{equation}
з матрицями $E - \lambda A$ та $E - \bar \lambda A^*$ відповідно і визначником $D(\lambda = |E - \lambda A| = |E - \bar \lambda A^*|$. \\

Дослідимо питання існування та єдиності розв'язку СЛАР (\ref{eq:2.8}) та (\ref{eq:2.8'}). \\

Нехай $D(\lambda) \ne 0$, $\rang |E - \lambda A| = \rang |E - \bar \lambda A^*| = N$, тоді СЛАР (\ref{eq:2.8}) і (\ref{eq:2.8}) мають єдиний розв’язок для будь-яких векторів $\vec a$ і $\vec b$ відповідно, а тому інтегральні рівняння Фредгольма (\ref{eq:2.2}), (\ref{eq:2.2'}) мають єдині розв’язки при будь-яких $f$ та $g$ відповідно, і ці розв’язки записуються за формулами (\ref{eq:2.3}), (\ref{eq:2.3'}). \\

Нехай $D(\lambda) = 0$, $\rang |E - \lambda A| = \rang |E - \bar \lambda A^*| = q < N$, тоді однорідні СЛАР 
\begin{equation}
	\label{eq:2.9}
	\vec c = \lambda A \vec c,
\end{equation}
та
\begin{equation}
	\label{eq:2.9'}
	\vec d = \lambda A^* \vec d,
\end{equation}
мають $N - q$ лінійно незалежних розв’язків $\vec c_s$, $\vec d_s$, $s = \overline{1, N - q}$, де вектор визначається формулою $\vec c_s = (c_{s1}, \ldots, c_{sN})$, $\vec d_s = (d_{s1}, \ldots, d_{sN})$, таким чином відповідні однорідні інтегральні рівняння Фредгольма рівнянням (\ref{eq:2.2}), (\ref{eq:2.2'}) мають $N - q$ лінійно незалежних розв’язків які записуються за такими формулами:
\begin{equation}
	\label{eq:2.10}
	\phi_s(x) = \lambda \Sum_{i = 1}^N c_{si} f_i(x), \quad s = \overline{1, N - q},
\end{equation}
\begin{equation}
	\label{eq:2.10'}
	\psi_s(x) = \bar \lambda \Sum_{i = 1}^N d_{si} \bar g_i(x), \quad s = \overline{1, N - q},
\end{equation}
$\phi_s(x)$, $\psi_s(x)$ -- власні функції, а число $N - q$ -- кратність характеристичного числа $\lambda$ та $\bar \lambda$. Кожна з систем функцій $\phi_s$, $\psi_s$, $s = \overline{1, N - q}$ лінійно незалежна, оскільки лінійно незалежними є системи функцій $f_i$ та $g_i$ і лінійно незалежні вектори $\vec c_s$ і $\vec d_s$, $s = \overline{1, N - q}$. \\

Нагадаємо одне з формулювань теореми Кронекера-Капеллі. Для існування розв’язку системи лінійних алгебраїчних рівнянь необхідно і достатньо що би вільний член рівняння був ортогональним всім розв’язкам спряженого однорідного рівняння. \\

Для нашого випадку цю умову можна записати у вигляді
\begin{equation}
	\label{eq:2.11}
	(\vec a, \vec d_s) = \Sum_{i = 1}^N a_i \bar d_{si} = 0, \quad \forall s = \overline{1, N - q}.
\end{equation}

Покажемо, що для виконання умови $(\vec a, \vec d_s) = 0$, $s = \overline{1, N - q}$ необхідно і достатньо, щоб вільний член інтегрального рівняння Фредгольма (\ref{eq:2.2}) був ортогональним розв'язкам спряженого однорідного рівняння тобто 
\begin{equation}
	\label{eq:2.12}
	(f, \psi_s) = 0, \quad s = \overline{1, N - q}
\end{equation}
Дійсно, з (\ref{eq:2.10'}) та (\ref{eq:2.4}) маємо:
\[ (f, \psi_s) = \Int_G f(x) \bar \psi_s (x) \diff x = \lambda \Sum_{i = 1}^N \bar d_{si} \Int_G f(x) g_i(x) \diff x = \lambda \Sum_{i = 1}^N a_i \bar d_{si} = \lambda (\vec a, \vec d_s) = 0, \] для всіх $s = \overline{1, N - q}$. \\

В цьому випадку розв'язок СЛАР не єдиний, і визначається з точністю до довільного розв'язку однорідної системи рівнянь, тобто з точністю до лінійної оболонки натягнутої на систему власних векторів характеристичного числа $\lambda$:
\begin{equation}
	\label{eq:2.13}
	\vec c = \vec c_0 + \Sum_{i = 1}^{N - q} \gamma_i \vec c_i,
\end{equation}
де $\gamma_i$ -- довільні константи, $\vec c_0$ -- будь-який розв'язок неоднорідної системи рівнянь $\vec c_0 = \lambda A \vec c_0 + \vec a$, тоді розв'язок інтегрального рівняння можна записати у вигляді:
\begin{equation}
	\label{eq:2.14}
	\phi(x) = \phi_0(x) + \Sum_{i = 1}^{N - q} \gamma_i \phi_i(x),
\end{equation}
де $\phi_0$ -- довільний розв'язок неоднорідного рівняння $\phi_0 = \lambda \bf{K} \phi_0 + f$. \\

Отже доведені такі теореми:

\begin{theorem}[Перша теорема Фредгольма для вироджених ядер]
	Якщо $D(\lambda) \ne 0$, то інтегральне рівняння (\ref{eq:2.2}) та спряжене до нього (\ref{eq:2.2'}) мають єдині розв'язки для довільних вільних членів $f$ та $g$ з класу неперервних функцій.
\end{theorem}
\begin{theorem}[Друга теорема Фредгольма для вироджених ядер]
	Якщо $D(\lambda) = 0$, то однорідне рівняння Фредгольма другого роду (\ref{eq:2.2}) ($f \equiv 0$) і спряжене до нього (\ref{eq:2.2'}) ($g \equiv 0$) мають однакову кількість лінійно незалежних розв'язків рівну $N - q$, де $q = \rang(E - \lambda A)$.
\end{theorem}
\begin{theorem}[Третя теорема Фредгольма для вироджених ядер]
	Якщо $D(\lambda) = 0$, то для існування розв'язків рівняння (\ref{eq:2.2}) необхідно і достатньо, щоб вільний член $f$ був ортогональним усім розв'язкам однорідного спряженого рівняння (\ref{eq:2.12}). При виконанні цієї умови розв'язок існує та не єдиний і визначається з точністю до лінійної оболонки натягнутої на систему власних функцій характеристичного числа $\lambda$.
\end{theorem}
\begin{corollary}
	Характеристичні числа виродженого ядра $K(x, y)$ співпадають з коренями поліному $D(\lambda) = 0$, а їх кількість не перевищує $N$.
\end{corollary}
\begin{example}
	Знайти розв’язок інтегрального рівняння \[ \phi(x) = \lambda \Int_0^\pi \sin(x - y) \phi(y) \diff y + \cos(x). \]
\end{example}
\begin{solution*}
	\[ \phi(x) = \lambda \sin(x) \Int_0^\pi \cos(y) \phi(y) \diff y - \lambda \cos(x) \Int_0^\pi \sin(y) \phi(y) \diff y + \cos(x). \]
	Позначимо \[ c_1 = \Int_0^\pi \cos(y) \phi(y) \diff y, \quad c_2 = \Int_0^\pi \sin(y) \phi(y) \diff y. \]
	\[ \phi(x) = \lambda (c_1 \sin(x) - c_2 \cos(x)) + \cos(x). \]
	Підставляючи останню рівність в попередні отримаємо систем рівнянь:
	\begin{system*}
		c_1 &= \Int_0^\pi \cos(y) (\lambda c_1 \sin(y) - \lambda c_2 \cos(y) + \cos(y)) \diff y, \\
		c_2 &= \Int_0^\pi \sin(y) (\lambda c_1 \sin(y) - \lambda c_2 \cos(y) + \cos(y)) \diff y.
	\end{system*}
	Після обчислення інтегралів:
	\begin{system*}
		c_1 + \frac{\lambda \pi}{2} c_2 &= \frac{\pi}{2}, \\
		- \frac{\lambda\pi}{2} c_1 + c_2 &= 0.
	\end{system*}
	Визначник цієї системи
	\[ D(\lambda) = \begin{vmatrix} 1 & \frac{\lambda\pi}{2} \\ -\frac{\lambda\pi}{2} & 1 \end{vmatrix} = 1 + \left( \frac{\lambda \pi}{2} \right)^2 \ne 0. \]
	За правилом Крамера маємо
	\[ c_1 = \dfrac{2 \pi}{4 + (\lambda \pi)^2}, \quad c_2 = \dfrac{\lambda \pi^2}{4 + (\lambda \pi)^2}. \]
	Таким чином розв’язок має вигляд
	\[ \phi(x) = \dfrac{2 \lambda \pi \sin(x) + 4 \cos (x)}{4 + (\lambda \pi)^2}. \]
\end{solution*}