\documentclass[a4paper, 12pt]{article}
\usepackage[utf8]{inputenc}
\usepackage[english, ukrainian]{babel}
\usepackage{amsmath, amssymb}
\usepackage[top = 2 cm, left = 1 cm, right = 1 cm, bottom = 2 cm]{geometry} 

\usepackage{fancyhdr}
\pagestyle{fancy}
\rhead{Нікіта Скибицький, ОМ-3}
\cfoot{\thepage}

\usepackage{multicol, graphicx}

\usepackage{amsthm}
\theoremstyle{definition}
\newtheorem{theorem}{Теорема}[subsection]
\newtheorem{definition}{Визначення}
\newtheorem{prove}{Доведення}
\newtheorem{problem}{\normalfont{\textit{Задача}}}[section]
\newtheorem*{solution}{Розв'язок}
\newtheorem{side_comment}{Зауваження}

\allowdisplaybreaks
\setlength\parindent{0pt}

\newcommand{\argmax}{\arg\max}
\newcommand{\argmin}{\arg\min}

\newcommand{\dif}{\mathrm{d}}
\newcommand{\dydx}{\dfrac{\dif y}{\dif x}}
\newcommand{\dxdt}{\dfrac{\dif x}{\dif t}}
\newcommand{\dydt}{\dfrac{\dif y}{\dif t}}

\newcommand{\NN}{\mathbb{N}} 
\newcommand{\ZZ}{\mathbb{Z}}
\newcommand{\QQ}{\mathbb{Q}}
\newcommand{\RR}{\mathbb{R}}
\newcommand{\CC}{\mathbb{C}}

\newcommand{\const}{\text{const}}

\newcommand{\LaReF}[1]{(\ref{#1})}

\renewcommand{\epsilon}{\varepsilon}
\renewcommand{\phi}{\varphi}

\newcommand{\ws}{\text{ }}

\newcommand{\Max}{\displaystyle\max\limits}
\newcommand{\Min}{\displaystyle\min\limits}
\newcommand{\Sum}{\displaystyle\sum\limits}
\newcommand{\Int}{\displaystyle\int\limits}
\newcommand{\Prod}{\displaystyle\prod\limits}

\newenvironment{system}{\begin{equation}\left\{\begin{aligned}}{\end{aligned}\right.\end{equation}}
\newenvironment{system*}{\begin{equation*}\left\{\begin{aligned}}{\end{aligned}\right.\end{equation*}}

\usepackage{color}

\usepackage{xfrac}

\title{{\Huge МАТЕМАТИЧНА ФІЗИКА}}
\author{Скибицький Нікіта}
\date{\today}

\usepackage{amsthm}
\usepackage[dvipsnames]{xcolor}
\usepackage{thmtools}
\usepackage[framemethod=TikZ]{mdframed}

\theoremstyle{definition}
\mdfdefinestyle{mdbluebox}{%
	roundcorner = 10pt,
	linewidth=1pt,
	skipabove=12pt,
	innerbottommargin=9pt,
	skipbelow=2pt,
	nobreak=true,
	linecolor=blue,
	backgroundcolor=TealBlue!5,
}
\declaretheoremstyle[
	headfont=\sffamily\bfseries\color{MidnightBlue},
	mdframed={style=mdbluebox},
	headpunct={\\[3pt]},
	postheadspace={0pt}
]{thmbluebox}

\mdfdefinestyle{mdredbox}{%
	linewidth=0.5pt,
	skipabove=12pt,
	frametitleaboveskip=5pt,
	frametitlebelowskip=0pt,
	skipbelow=2pt,
	frametitlefont=\bfseries,
	innertopmargin=4pt,
	innerbottommargin=8pt,
	nobreak=true,
	linecolor=RawSienna,
	backgroundcolor=Salmon!5,
}
\declaretheoremstyle[
	headfont=\bfseries\color{RawSienna},
	mdframed={style=mdredbox},
	headpunct={\\[3pt]},
	postheadspace={0pt},
]{thmredbox}

\declaretheorem[style=thmbluebox,name=Теорема,numberwithin=subsection]{theorem}
\declaretheorem[style=thmbluebox,name=Лема,sibling=theorem]{lemma}
\declaretheorem[style=thmbluebox,name=Твердження,sibling=theorem]{proposition}
\declaretheorem[style=thmbluebox,name=Закон,sibling=theorem]{law}
\declaretheorem[style=thmbluebox,name=Рівняння,sibling=theorem]{th_equation}
\declaretheorem[style=thmbluebox,name=Наслідок,sibling=theorem]{corollary}

\declaretheorem[style=thmredbox,name=Приклад,sibling=theorem]{example}
\declaretheorem[style=thmredbox,name=Приклади,sibling=example]{examples}

\declaretheorem[style=thmredbox,name=Властивість,sibling=theorem]{property}
\declaretheorem[style=thmredbox,name=Властивості,sibling=property]{properties}

\mdfdefinestyle{mdgreenbox}{%
	skipabove=8pt,
	linewidth=2pt,
	rightline=false,
	leftline=true,
	topline=false,
	bottomline=false,
	linecolor=ForestGreen,
	backgroundcolor=ForestGreen!5,
}
\declaretheoremstyle[
	headfont=\bfseries\sffamily\color{ForestGreen!70!black},
	bodyfont=\normalfont,
	spaceabove=2pt,
	spacebelow=1pt,
	mdframed={style=mdgreenbox},
	headpunct={ --- },
]{thmgreenbox}

\mdfdefinestyle{mdblackbox}{%
	skipabove=8pt,
	linewidth=3pt,
	rightline=false,
	leftline=true,
	topline=false,
	bottomline=false,
	linecolor=black,
	backgroundcolor=RedViolet!5!gray!5,
}
\declaretheoremstyle[
	headfont=\bfseries,
	bodyfont=\normalfont\small,
	spaceabove=0pt,
	spacebelow=0pt,
	mdframed={style=mdblackbox}
]{thmblackbox}

\declaretheorem[name=Вправа,sibling=theorem,style=thmblackbox]{exercise}
\declaretheorem[name=Запитання,sibling=theorem,style=thmgreenbox]{ques}
\declaretheorem[name=Зауваження,sibling=theorem,style=thmgreenbox]{remark}
\declaretheorem[name=Визначення,sibling=theorem,style=thmblackbox]{definition}

\newtheorem{problem}{Задача}[subsection]
\newtheorem{sproblem}[problem]{Задача}
\newtheorem{dproblem}[problem]{Задача}
\renewcommand{\thesproblem}{\theproblem$^{\star}$}
\renewcommand{\thedproblem}{\theproblem$^{\dagger}$}
\newcommand{\listhack}{$\empty$\vspace{-2em}} 

\theoremstyle{remark}
\newtheorem*{solution}{Розв'язок}


\begin{document}

\tableofcontents

\setcounter{section}{4}
\setcounter{subsection}{3}
\setcounter{subsubsection}{4}
\setcounter{theorem}{21}
\setcounter{equation}{55}

\subsubsection{Функція Гріна граничних задач оператора теплопровідності}

Будемо розглядати граничні задачі для рівняння теплопровідності:

\begin{system}
	& a^2 \Delta u(x, t) - \frac{\partial u(x,t)}{\partial t} = - F(x, t), \\
	& u(x, 0) = u_0(x), \\
	& \left. \ell_i u(x, t) \right|_{x \in S} = f(x, t), \quad i = 1, 2, 3.
\end{system}
для $x \in \Omega$, $t > 0$. \medskip

Тут 
\begin{align}
	\left. \ell_1 u(x, t) \right|_{x \in S} &= \left. u(x, t) \right|_{x \in S}, \\
	\left. \ell_2 u(x, t) \right|_{x \in S} &= \left. \frac{\partial u(x, t)}{\partial n} \right|_{x \in S}, \\
	\left. \ell_3 u(x, t) \right|_{x \in S} &= \left. \frac{\partial u(x, t)}{\partial n} + \alpha(x, t) \cdot u(x, t) \right|_{x \in S}
\end{align}
--- оператори граничних умов першого, другого, або третього роду.

\begin{definition}[функції Гріна граничної задачі теплопровідності]
	Функцію $E_i (x, \xi, t - \tau)$ будемо називати \textit{функцією Гріна першої, другої або третьої граничної задачі рівняння теплопровідності} в області $\Omega$ з границею $S$ для $t > 0$, якщо вона є розв'язком настуної граничної задачі:
	\begin{system}
		& a^2 \Delta_x E_i (x, \xi, t - \tau) - \frac{\partial E_i(x, \xi, t - \tau)}{\partial t} = - \delta(x - \xi, t - \tau), \\
		& \left. E_i(x, \xi, t - \tau) \right|_{t - \tau \le 0} = 0, \\
		& \left. \ell_i E_i (x, \xi, t - \tau) \right|_{x \in S} = 0, \quad i = 1, 2, 3.
	\end{system}
	для $x \in \Omega$, $t > 0$
\end{definition}

Еквівалетнне визначення можна надати у вигляді
\begin{definition}[функції Гріна граничної задачі теплопровідності]
	Функцію $E_i (x, \xi, t - \tau)$ будемо називати \textit{функцією Гріна першої, другої або третьої граничної задачі рівняння теплопровідності} в області $\Omega$ з границею $S$ для $t > 0$, якщо вона може бути представлена у вигляді
	\begin{equation}
		E_i(x, \xi, t - \tau) = \epsilon(x - \xi, t - \tau) + \omega_i(x, \xi, t - \tau),
	\end{equation}
	де  перший доданок є фундаментальним розв'язком оператора теплопровідності, а другий  є розв'язком наступної граничної задачі
	\begin{system}
		& a^2 \Delta_x \omega_i (x, \xi, t - \tau) - \frac{\partial \omega_i(x, \xi, t - \tau)}{\partial t} = - \delta(x - \xi, t - \tau), \\
		& \left. \omega_i(x, \xi, t - \tau) \right|_{t - \tau \le 0} = 0, \\
		& \left. \ell_i \omega_i (x, \xi, t - \tau) \right|_{x \in S} = -\left.\ell_i \epsilon_i(x - \xi, t - \tau)\right|_{x \in S} \quad i = 1, 2, 3.
	\end{system}
	для $x \in \Omega$, $t > 0$.
\end{definition}

Вивчимо 
\begin{properties}[функції Гріна оператора теплопровідності]
	Легко бачити, що 
	\begin{enumerate}
		\item Функція Гріна граничних задач рівняння теплопровідності з аргументами $E_i(x, \xi, -t)$ задовольняє спряженому диференціальному рівнянню
		\begin{equation}
			a^2 \Delta_x E_i(x, \xi, -t) + \frac{\partial E_i(x, \xi, - t)}{\partial t} = - \delta(x - \xi) \delta (t - \tau), 
		\end{equation}
		для усіх $x, \xi \in \Omega$, і $t > 0$.
		\item Функція Гріна є також симетричною функцією своїх перших двох аргументів.
	\end{enumerate}
\end{properties}
	
\begin{proof}
	Доведемо другу властивість. Запишемо співвідношення, яким задовольняє функція Гріна:
	\begin{align}
		a^2 \Delta_x E_i(x, \xi, t - \tau_1) + \frac{\partial E_i(x, \xi, t - \tau_1)}{\partial t} &= - \delta(x - \xi) \delta (t - \tau_1), \quad x, \xi \in \Omega, \\
		a^2 \Delta_x E_i(x, \eta, \tau_2 - t) + \frac{\partial E_i(x, \eta, \tau_2 - t)}{\partial t} &= - \delta(x - \eta) \delta (t - \tau_2), \quad x, \eta \in \Omega.
	\end{align}
	Перше рівняння помножимо на $E_i(x, \xi, \tau_2 - t)$, друге рівняння помножимо на $E_i(x, \xi, t - \tau_1)$, віднімемо від першого друге і проінтегруємо по $x \in \Omega$ і по $-\infty < t < \tau$:
	\begin{equation}
		\begin{aligned}
			& a^2 \Iiint_\Omega \Int_{-\infty}^\tau \Big( E_i(x, \eta, \tau_2 - t) \Delta_x E_i(x, \xi, t - \tau_1) - \\
			& \qquad \quad - E_i(x, \xi, t - \tau_1) \Delta_x E_i(x, \eta, \tau_2 - t) \Big) \diff t \diff x - \\
			& \qquad - \Iiint_\Omega \Int_{-\infty}^\tau \frac{\partial}{\partial \tau} \left( E_i(x, \xi, t - \tau_1) E_i(x, \eta, \tau_2 - t) \right) \diff t \diff x = \\
			& \quad = - E_i(\xi, \eta, \tau_2 - \tau_1) + E_i(\eta, \xi, \tau_2 - \tau_1).
		\end{aligned}
	\end{equation}

	В результаті застосування другої формули Гріна до першого інтегралу в лівій частині рівності і обчислення другого інтегралу лівої частини, отримаємо:
	\begin{equation}
		\begin{aligned}
			& E_i(\xi, \eta, \tau_2 - \tau_1 ) - E_i(\eta, \xi, \tau_2 - \tau_1) = \\
			& \quad = - \Iiint_\Omega \Big( E_i(x, \xi, \tau - \tau_1) E_i(x, \eta, \tau_2 - \tau) - \\
			& \qquad \quad - E_i(x, \xi, -\infty) E_i(x, \eta, -\infty) \Big) \diff \Omega + \\
			& \qquad + a^2 \Iint_S \Int_{-\infty}^\tau \left( \frac{\partial E_i(x, \xi, t - \tau_1)}{\partial n} E_i(x, \eta, \tau_2 - t) \right. - \\
			& \qquad \quad - \left. E_i(x, \xi, t - \tau_1) \frac{\partial E_i(x, \eta, \tau_2 - t)}{\partial n} \right) \diff S \diff t.
		\end{aligned}
	\end{equation}

	Обираючи $\tau > \tau_2 > \tau_1$ , отримаємо з урахування граничних і початкових умов для функції Гріна, що інтеграли в правій частині останньої рівності дорівнює нулю.
\end{proof}

Для отримання інтегрального представлення розв'язків граничних задач, запишемо граничну задачу теплопровідності в змінних $\xi$, $\tau$ :
\begin{system}
	& a^2 \Delta u(\xi, \tau) - \frac{\partial u(\xi, \tau)}{\partial \tau} = - F(\xi, \tau), \quad \xi \in \Omega, \quad \tau > 0, \\
	& u(\xi, 0) = u_0(\xi), \\
	& \left. \ell_i u(\xi, \tau) \right|_{\xi \in S} = f(\xi, \tau), \quad i = 1, 2, 3. 
\end{system}
та рівняння для функції Гріна по змінних $\xi$, $\tau$:
\begin{equation}
	a^2 \Delta_\xi E_i(x, \xi, t - \tau) + \frac{\partial E_i(x, \xi, t - \tau)}{\partial \tau} = - \delta(x - \xi) \delta(t - \tau),
\end{equation}
де $x, \xi \in \Omega$, $ t > \tau > 0$. \medskip

Перше рівняння помножимо на $E_i(x, \xi, t - \tau)$, а друге --- на $u(\xi, \tau$, віднімемо від першого друге, і проінтегруємо по $0 < \tau < t + \epsilon$ та по $\xi \in \Omega$. \medskip

Отриаємо співвідношення:
\begin{equation}
	\begin{aligned}
		& a^2 \Int_0^{t + \epsilon} \Iiint_\Omega \Big( E_i(x, \xi, t - \tau) \Delta u(\xi, \tau) - \\
		& \qquad \quad - u(\xi, \tau) \Delta_\xi E_i(x, \xi, t - \tau) \Big) \diff \xi \diff \tau + \\
		& \qquad + \Int_0^{t + \epsilon} \Iiint_\Omega E_i(x, \xi, t - \tau) F(\xi, \tau) \diff \xi \diff \tau - \\
		& \qquad -  \Iiint_\Omega \Int_0^{t+\epsilon} \frac{\partial (E_u(x, \xi, t - \tau) u(\xi, \tau))}{\partial \tau} \diff \tau \diff \xi = \\
		& \quad =  \Int_0^{t + \epsilon} \Iiint_\Omega \delta(x - \xi) \delta(t - \tau) \diff \xi \diff \tau.
	\end{aligned}
\end{equation}

Після застосування другої формули Гріна до першого інтегралу, обчислення третього інтегралу при $\epsilon \to 0$ отримаємо наступну проміжну формулу:
\begin{equation}
	\begin{aligned}
		u(x, t) &= \Int_0^t \Iiint_\Omega E_i(x, \xi, t - \tau) F(\xi, \tau) \diff \xi \diff \tau + \\
		& \quad + \Iiint_\Omega E_i(x, \xi, t) u_0(\xi) \diff \xi + \\
		& \quad + a^2 \Int_0^t \Iint_S \left( E_i(x, \xi, t - \tau) \frac{\partial u(\xi, \tau)}{\partial n_\xi} - u(\xi, \tau) \frac{\partial E_i(x, \xi, t - \tau)}{\partial n_\xi} \right) \diff S_\xi \diff \tau.
	\end{aligned}
\end{equation}

Враховуючи відповідні граничні умови, яким задовольняє розв'язок на границі поверхні $S$ отримаємо для першої граничної задачі:
\begin{equation}
	\begin{aligned}
		u(x, t) &= \Int_0^t \Iiint_\Omega E_1(x, \xi, t - \tau) F(\xi, \tau) \diff \xi \diff \tau + \\
		& \quad + \Iiint_\Omega E_1(x, \xi, t) u_0(\xi) \diff \xi - \\
		& \quad - a^2 \Int_0^t \Iint_S \left( \frac{\partial E_1(x, \xi, t - \tau)}{\partial n_\xi} f(\xi, \tau)\right) \diff S_\xi \diff \tau.
	\end{aligned}
\end{equation}

Для другої та третьої граничних задач отримаємо 
\begin{equation}
	\begin{aligned}
		u(x, t) &= \Int_0^t \Iiint_\Omega E_i(x, \xi, t - \tau) F(\xi, \tau) \diff \xi \diff \tau + \\
		& \quad + \Iiint_\Omega E_i(x, \xi, t) u_0(\xi) \diff \xi + \\
		& \quad + a^2 \Int_0^t \Iint_S E_i(x, \xi, t - \tau) f(\xi, \tau) \diff S_\xi \diff \tau.
	\end{aligned}
\end{equation}

\subsubsection{Функція Гріна граничних задач хвильового оператора}

\end{document}