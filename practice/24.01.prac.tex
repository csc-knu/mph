% cd ..\..\Users\NikitaSkybytskyi\Desktop\c3s2\mathematical-physics
% pdflatex 24.01.prac.tex && pdflatex 24.01.prac.tex && del 24.01.prac.out, 24.01.prac.log, 24.01.prac.aux, 24.01.prac.toc && start 24.01.prac.pdf

\documentclass[a4paper, 12pt]{article}
\usepackage[utf8]{inputenc}
\usepackage[english, ukrainian]{babel}
\usepackage{amsmath, amssymb}
\usepackage[top = 2 cm, left = 1 cm, right = 1 cm, bottom = 2 cm]{geometry} 

\usepackage{fancyhdr}
\pagestyle{fancy}
\rhead{Нікіта Скибицький, ОМ-3}
\cfoot{\thepage}

\usepackage{multicol, graphicx}

\usepackage{amsthm}
\theoremstyle{definition}
\newtheorem{theorem}{Теорема}[subsection]
\newtheorem{definition}{Визначення}
\newtheorem{prove}{Доведення}
\newtheorem{problem}{\normalfont{\textit{Задача}}}[section]
\newtheorem*{solution}{Розв'язок}
\newtheorem{side_comment}{Зауваження}

\allowdisplaybreaks
\setlength\parindent{0pt}

\newcommand{\argmax}{\arg\max}
\newcommand{\argmin}{\arg\min}

\newcommand{\dif}{\mathrm{d}}
\newcommand{\dydx}{\dfrac{\dif y}{\dif x}}
\newcommand{\dxdt}{\dfrac{\dif x}{\dif t}}
\newcommand{\dydt}{\dfrac{\dif y}{\dif t}}

\newcommand{\NN}{\mathbb{N}} 
\newcommand{\ZZ}{\mathbb{Z}}
\newcommand{\QQ}{\mathbb{Q}}
\newcommand{\RR}{\mathbb{R}}
\newcommand{\CC}{\mathbb{C}}

\newcommand{\const}{\text{const}}

\newcommand{\LaReF}[1]{(\ref{#1})}

\renewcommand{\epsilon}{\varepsilon}
\renewcommand{\phi}{\varphi}

\newcommand{\ws}{\text{ }}

\newcommand{\Max}{\displaystyle\max\limits}
\newcommand{\Min}{\displaystyle\min\limits}
\newcommand{\Sum}{\displaystyle\sum\limits}
\newcommand{\Int}{\displaystyle\int\limits}
\newcommand{\Prod}{\displaystyle\prod\limits}

\newenvironment{system}{\begin{equation}\left\{\begin{aligned}}{\end{aligned}\right.\end{equation}}
\newenvironment{system*}{\begin{equation*}\left\{\begin{aligned}}{\end{aligned}\right.\end{equation*}}

\usepackage{color}

\usepackage{xfrac}

\title{Математична фізика::практика}
\author{Нікіта Скибицький}
\date{\today}

\begin{document}

\maketitle

\tableofcontents

\section{24.01. Метод Фур'є}

Метод також називають \textit{методом розділення змінних}.

\subsection{Теорія}

Розглянемо наступну постановку задачі:

\begin{problem*}
	Знайти вільні коливання струни довжини $\ell$ із закріпленими кінцями у середовищі без опору. Початкове положення струни і її швдикість задані. \\

	Цю задачу можна формалізувати наступним чином:
	\begin{equation}
		\label{eq:1}
		\frac{\partial^2 u}{\partial t^2} = a^2 \cdot \frac{\partial^2 u}{\partial x^2}, \quad x \in (0, \ell), \quad t > 0,
	\end{equation}
	закріплені кінці (крайові умови):
	\begin{equation}
		\label{eq:2}
		u(0, t) = 0 = u(\ell, t),
	\end{equation}
	початкові\footnote{Тут $u_t$ позначає $\frac{\partial u}{\partial t}$.} умови:
	\begin{align}
		\label{eq:3}
		u(x, 0) &= \phi(x), \quad x \in [0, \ell], \\
		\label{eq:4}
		u_t(x, 0) &= \psi(x), \quad x \in [0, \ell).
	\end{align}
\end{problem*}

Метод Фур'є застосовується до задач, у яких саме рівняння \eqref{eq:1} та крайові умови \eqref{eq:2} є однорідними (по $x$).

\begin{solution}
	Будемо шукати розв'язок у вигляді
	\begin{equation}
		\label{eq:5}
		u(x, t) = X(x) \cdot T(t) \not\equiv 0.
	\end{equation}

	Підставляємо \eqref{eq:5} в \eqref{eq:1}:
	\begin{equation*}
		X(x) \cdot T''(t) = a^2 \cdot X''(x) \cdot T(t).
	\end{equation*}

	Ділимо\footnote{Розв'язок нетривіальний тому можемо собі таке дозволити.} обидві частини на $a^2 \cdot X(x) \cdot T(t)$ маємо:
	\begin{equation*}
		\frac{T''(t)}{a^2 \cdot T(t)} = \frac{X''(x)}{X(x)}.
	\end{equation*}

	Оскільки ліва частина не залежить від $x$ а права від $t$, то робимо висновок що вони обидві -- сталі. Позначимо відповідну сталу через $-\lambda$, отримаємо:
	\begin{equation*}
		\frac{T''(t)}{a^2 \cdot T(t)} = \frac{X''(x)}{X(x)} = - \lambda.
	\end{equation*}

	Звідси маємо два рівняння:
	\begin{align*}
		T''(t) + \lambda \cdot a^2 \cdot T(t) &= 0, \\
		X''(x) + \lambda \cdot X(x) &= 0.
	\end{align*}

	Поставимо тепер задачу Штурма-Ліувілля для функції $X(x)$. \\

	З крайових умов \eqref{eq:2} маємо:
	\begin{equation*}
		u(0, t) = X(0) \cdot T(t) = 0.
	\end{equation*}

	Оскільки $T \not\equiv 0$, то звідси $X(0) = 0$. Аналогічно $X(\ell) = 0$. \\

	Нагадаємо \textit{постановку задачі Штурма-Ліувілля}:

	\begin{problem*}
		Необхідно знайти усі $\lambda$ для яких задане однорідне рівняння має нетривіальний розв'язок.
	\end{problem*}

	Знайдемо $\lambda$ з характеристичного рівняння $\kappa^2 = - \lambda$, маємо три випадки:

	\begin{enumerate}
		\item $\lambda = 0$. Тоді розв'язок рівняння має вигляд\footnote{Тут $a$ -- не параметр початквої задачі а локальний невідомий коефіцієнт.} $X(x) = a x + b$, підставляємо його в крайові умови:
		\begin{align*}
			0 &= X(0) = b, \\
			0 &= X(\ell) = a \ell + b.
		\end{align*}

		Нескладно бачити, що $a = b = 0$, тобто отримали тривіальний розв'язок $X \equiv 0$ який нас не влаштовує.
		\item $\lambda < 0$, тоді розв'язок рівняня має вигляд\footnote{Тут $a$ -- не параметр початквої задачі а локальний невідомий коефіцієнт.} 
		\begin{equation*}
			X(x) = a \cdot e^{\sqrt{-\lambda} \cdot x} + b \cdot e^{-\sqrt{-\lambda} \cdot x},
		\end{equation*}
		підставляємо його в крайові умови:
		\begin{align*}
			0 &= X(0) = a + b = 0, \\
			0 &= X(\ell) = a \cdot e^{\sqrt{-\lambda} \cdot \ell} + b \cdot e^{-\sqrt{-\lambda} \cdot \ell}.
		\end{align*}

		Нескладно бачити, що у цієї системи немає нетривіальни розв'язків, зокрема тому що її визначник $\ne$ 0.
		\item $\lambda > 0$, тоді розв'язок рівняня має вигляд\footnote{Тут $a$ -- не параметр початквої задачі а локальний невідомий коефіцієнт.} 
		\begin{equation*}
			X(x) = a \cdot \cos \left( \sqrt{\lambda} \cdot x \right) + b \cdot \sin \left( \sqrt{\lambda} \cdot x \right),
		\end{equation*}
		підставляємо його в крайові умови:
		\begin{align*}
			0 &= X(0) = a, \\
			0 &= X(\ell) = b \sin \left( \sqrt{\lambda} \ell \right).
		\end{align*}

		Звідси $\lambda_n = \left( \frac{\pi n}{\ell} \right)^2$, $n \in \NN$. \\
	\end{enumerate}

	Відповідний розв'язок набуває вигляду
	\begin{equation*}
		X_n(x) = \sin \left( \frac{\pi n x}{\ell} \right), \quad n \in \NN.
	\end{equation*}

	Пісдтавляємо знайдене $\lambda_n$ у рівняння на $T$:
	\begin{equation*}
		T_n''(t) + \left( \frac{\pi n}{\ell} \right)^2 \cdot T_n(t) = 0.
	\end{equation*}

	Тут параметр вже цілком конкретний, тому можемо одразу записати загальний вигляд розв'язку:
	\begin{equation*}
		T_n(t) = a_n \cdot \cos \left( \frac{\pi n a t}{\ell} \right) + b_n \cdot \sin \left( \frac{\pi n a t}{\ell} \right).
	\end{equation*}
	
	З \eqref{eq:5} маємо $u_n(x, t) = X_n(x) \cdot T_n(t)$. Але всі ці розв'язки лінійно незалежні, тому загальним розв'язком буде
	\begin{multline}
		\label{eq:6}
		u(x, t) = \Sum_{n = 1}^\infty X_n(x) \cdot T_n(t) = \\ = \Sum_{n = 1}^\infty \sin \left( \frac{\pi n x}{\ell} \right) \cdot \left( a_n \cdot \cos \left( \frac{\pi n a t}{\ell} \right) + b_n \cdot \sin \left( \frac{\pi n a t}{\ell} \right) \right).
	\end{multline}

	Залишилося знайти $a_n$, $b_n$ з умов \eqref{eq:3} та \eqref{eq:4}. \\

	Підставляємо \eqref{eq:6} в \eqref{eq:3}:
	\begin{equation*}
		u(x, 0) = \Sum_{n = 1}^\infty a_n \cdot \sin \left( \frac{\pi n x}{\ell} \right) = \phi(x) = \ldots
	\end{equation*}

	\begin{remark*}
		Тут ми припускаємо, що $\phi(x)$ відповідає умовам теореми Стєклова і граничним умовам нашої задачі Штурма-Ліувілля.
	\end{remark*}

	\begin{equation*}
		\ldots = \Sum_{n = 1}^\infty \phi_n \cdot X_n(x) = \Sum_{n = 1}^\infty \phi_n \cdot \sin \left( \frac{\pi n x}{\ell} \right).
	\end{equation*}

	Прирівнюючи коефіцієнти\footnote{$\phi_n = \frac{1}{\|X_n\|^2} \cdot \int_0^\ell \phi(x) \cdot X_n(x) \cdot \diff x$.} в цих рядах Фур'є знаходимо що $a_n = \phi_n$.\\

	Знайдемо\footnote{Тут $\psi_n$ визначається аналогічно до $\phi_n$ вище.} $b_n$ з \eqref{eq:4}:

	\begin{equation*}
		u_t(x, 0) = \Sum_{n = 1}^\infty \sin \left( \frac{\pi n x}{\ell} \right) \cdot b_n \cdot \frac{\pi n a}{\ell} = \psi(x) = \Sum_{n = 1}^\infty \psi_n \cdot \sin \left( \frac{\pi n x}{\ell} \right).
	\end{equation*}

	Прирівнюючи коефіцієнти в цих рядах Фур'є знаходимо що $b_n = \psi_n \cdot \frac{\ell}{\pi n a}$.\\	

	Далі підставляємо $a_n, b_n$ в \eqref{eq:6}.
\end{solution}

Ми вже не будемо тут цього робити, оскілкьи подальші формули в загальному випадку доволі громізкі, і не дуже змістовні. \\

Рівняння коливання струни є рівнянням гіперболічного типу. Метод Фур'є можна також застосовувати до рівнянь параболічного типу, таких як розводіл тепла:

\begin{problem*}
	Знайти розводіл температури теплоізольованого (немає стоків) стержня на кінцях якого підтримується нульова температура. Початковий розподіл температури заданий. \\

	Формально, маємо рівняння
	\begin{equation}
		\label{eq:1-parabolic}
		\frac{\partial u}{\partial t} = a^2 \cdot \frac{\partial^2 u}{\partial x^2}
	\end{equation}
	на кінцях підтримуєтсья нульова температура:
	\begin{equation}
		\label{eq:2-parabolic}
		u(0, t) = 0 = u(\ell, t),
	\end{equation}
	початкові умови:
	\begin{align}
		\label{eq:3-parabolic}
		u(x, 0) &= \phi(x), \quad x \in [0, \ell].
	\end{align}
\end{problem*}

\begin{solution}
	Будемо шукати розв'язок у вигляді \eqref{eq:5}.

	Підставляємо \eqref{eq:5} в \eqref{eq:1}:
	\begin{equation*}
		X(x) \cdot T'(t) = a^2 \cdot X''(x) \cdot T(t).
	\end{equation*}

	Ділимо обидві частини на $a^2 \cdot X(x) \cdot T(t)$ маємо:
	\begin{equation*}
		\frac{T(t)}{a^2 \cdot T(t)} = \frac{X''(x)}{X(x)}.
	\end{equation*}

	Звідси маємо два рівняння:
	\begin{align*}
		T'(t) + \lambda \cdot a^2 \cdot T(t) &= 0, \\
		X''(x) + \lambda \cdot X(x) &= 0.
	\end{align*}

	Задача Штурма-Ліувілля для функції $X(x)$ вже розв'язана, одразу переходимо до знаходження $T(t)$. \\

	Пісдтавляємо знайдене $\lambda_n$ у рівняння на $T$:
	\begin{equation*}
		T_n'(t) + \left( \frac{\pi n}{\ell} \right)^2 \cdot T_n(t) = 0.
	\end{equation*}

	Тут параметр вже цілком конкретний, тому можемо одразу записати загальний вигляд розв'язку:
	\begin{equation*}
		T_n(t) = a_n \cdot \exp \left\{ - \left( \frac{\pi n a}{\ell} \cdot t\right) \right\}.
	\end{equation*}
	
	Загальним розв'язком буде
	\begin{equation}
		\label{eq:6-parabolic}
		u(x, t) = \Sum_{n = 1}^\infty \sin \left( \frac{\pi n x}{\ell} \right) \cdot a_n \cdot \exp \left\{ - \left( \frac{\pi n a}{\ell} \cdot t\right) \right\}.
	\end{equation}

	Залишилося знайти $a_n$ з умови \eqref{eq:3-parabolic}. \\

	Підставляємо \eqref{eq:6-parabolic} в \eqref{eq:3-parabolic}:
	\begin{multline*}
		u(x, 0) = \Sum_{n = 1}^\infty a_n \cdot \sin \left( \frac{\pi n x}{\ell} \right) = \\ = \phi(x) = \Sum_{n = 1}^\infty \phi_n \cdot X_n(x) = \Sum_{n = 1}^\infty \phi_n \cdot \sin \left( \frac{\pi n x}{\ell} \right).
	\end{multline*}

	Прирівнюючи коефіцієнти в цих рядах Фур'є знаходимо що $a_n = \phi_n$.\\

	Далі підставляємо $a_n$ в \eqref{eq:6-parabolic}.
\end{solution}

\subsection{Практика}

\begin{problem}[Владіміров, №20.14.1]
	\begin{equation*}
		u_{tt} = u_{xx} - 4 u, \quad 0 < x < 1
	\end{equation*}
	\begin{equation*}
		u|_{x = 0} = u|_{x = 1} = 0
	\end{equation*}
	\begin{equation*}
		u|_{t = 0} = x^2 - x
	\end{equation*}
	\begin{equation*}
		u_t|_{t = 0} = 0
	\end{equation*}
\end{problem}

\begin{solution}
	Будемо шукати розв'язок у вигляді
	\begin{equation*}
		u(t, x) = T(t) \cdot X(x).
	\end{equation*}

	Підставляємо
	\begin{equation*}
		T''(t) \cdot X(x) = T(T) \cdot X''(x) - 4 \cdot T(t) \cdot X(x).
	\end{equation*}
	
	Ділимо обидві частини на $X(x) \cdot T(t)$, маємо:
	\begin{equation*}
		\frac{T''(t)}{T(t)} = \frac{X''(x)}{X(x)} - 4 = - \lambda.
	\end{equation*}
	
	Поставимо тепер задачу Штурма-Ліувілля для функції $X(x)$:
	\begin{equation*}
		X''(x) + (\lambda - 4) \cdot X(x) = 0.
	\end{equation*}

	Знайдемо $\lambda$ з характеристичного рівняння $\kappa^2 = 4 - \lambda$, маємо три випадки:
	\begin{enumerate}
		\item $\lambda - 4 = 0 \implies X \equiv 0$;
		\item $\lambda - 4 < 0 \implies X \equiv 0$;
		\item $\lambda - 4 > 0 \implies \lambda_n = (\pi n)^2 + 4$, $n \in \NN$.
	\end{enumerate}

	Відповідний розв'язок набуває вигляду
	\begin{equation*}
		X_n(x) = \sin (\pi n x), \quad n \in \NN.
	\end{equation*}

	Пісдтавляємо знайдене $\lambda_n$ у рівняння на $T$:
	\begin{equation*}
		T''(t) + ((\pi n)^2 + 4) T(t) = 0.
	\end{equation*}

	Тут параметр вже цілком конкретний, тому можемо одразу записати загальний вигляд розв'язку:
	\begin{equation*}
		T_n(t) = a_n \cdot \cos \left( \sqrt{(\pi n)^2 + 4} \cdot t \right) + b_n \cdot \sin \left( \sqrt{(\pi n)^2 + 4} \cdot t \right).
	\end{equation*}

	Загальним розв'язком буде
	\begin{equation*}
		u(x, t) = \Sum_{n = 1}^\infty \sin(\pi n x) \cdot \left( a_n \cdot \cos \left( \sqrt{(\pi n)^2 + 4} \cdot t \right) + b_n \cdot \sin \left( \sqrt{(\pi n)^2 + 4} \cdot t \right) \right).
	\end{equation*}

	Знайдемо $a_n$ з
	\begin{equation*}
		\Sum_{n = 1}^\infty a_n \cdot \sin(\pi n x) = u|_{t = 0} = x^2 - x = \phi(x) = \Sum_{n = 1}^\infty \phi_n \cdot \sin(\pi n x).
	\end{equation*}

	Після певних неприємних обчислень маємо:
	\begin{equation*}
		a_n = \phi_n = \begin{cases} -\frac{8}{(\pi n)^3}, & n = 2 k - 1, \\ 0, & n = 2 k. \end{cases}
	\end{equation*}

	Знайдемо $b_n$ з
	\begin{equation*}
		\Sum_{n = 1}^\infty b_n \cdot \sin(\pi n x) \cdot \sqrt{(\pi n)^2 + 4} = u_t|_{t = 0} = 0.
	\end{equation*}
	
	Звідси одразу маємо $b_n = 0$, зокрема тому, що функції $\sin (\pi n x)$ є лінійно-незалежними.
\end{solution}

\subsection{Домашнє завдання}

\begin{problem}[Владіміров, №20.41.1]
	Розв'язати змішану задачу:
	\begin{equation*}
		u_t = u_{xx}, \quad 0 < x < 1	
	\end{equation*}
	\begin{equation*}
		u_x|_{x = 0} = 0 = u_x|_{x = 1}
	\end{equation*}
	\begin{equation*}
		u|_{t = 0} = x^2 - 1
	\end{equation*}
\end{problem}

\begin{solution}
	Будемо шукати розв'язок у вигляді
	\begin{equation*}
		u(t, x) = T(t) \cdot X(x).
	\end{equation*}

	Підставляємо
	\begin{equation*}
		T'(t) \cdot X(x) = T(T) \cdot X''(x).
	\end{equation*}
	
	Ділимо обидві частини на $X(x) \cdot T(t)$, маємо:
	\begin{equation*}
		\frac{T'(t)}{T(t)} = \frac{X''(x)}{X(x)} = - \lambda.
	\end{equation*}
	
	Поставимо тепер задачу Штурма-Ліувілля для функції $X(x)$:
	\begin{equation*}
		X''(x) + \lambda \cdot X(x) = 0.
	\end{equation*}

	Її крайові умови:
	\begin{equation*}
		X'(0) = 0 = X'(1)
	\end{equation*}

	Знайдемо $\lambda$ з характеристичного рівняння $\kappa^2 = - \lambda$, маємо три випадки:
	\begin{enumerate}
		\item $\lambda = 0 \implies X \equiv 1$;
		\item $\lambda < 0 \implies X \equiv 0$;
		\item $\lambda > 0 \implies \lambda_n = (\pi n)^2$, $n \in \NN$.
	\end{enumerate}

	Відповідний розв'язок набуває вигляду
	\begin{equation*}
		X_n(x) = \cos(\pi n x), \quad n \in \NN.
	\end{equation*}

	Пісдтавляємо знайдене $\lambda_n$ у рівняння на $T$:
	\begin{equation*}
		T'(t) + (\pi n)^2 \cdot T(t) = 0.
	\end{equation*}

	Тут параметр вже цілком конкретний, тому можемо одразу записати загальний вигляд розв'язку:
	\begin{equation*}
		T_n(t) = a_n \cdot e^{ - (\pi n)^2 t}.
	\end{equation*}

	Загальним розв'язком буде
	\begin{equation*}
		u(x, t) = \Sum_{n = 1}^\infty \cos(\pi n x) \cdot a_n \cdot e^{ - (\pi n)^2 t}.
	\end{equation*}

	Знайдемо $a_n$ з
	\begin{equation*}
		\Sum_{n = 1}^\infty a_n \cdot \cos(\pi n x) = u|_{t = 0} = x^2 - 1 = \phi(x) = \Sum_{n = 1}^\infty \phi_n \cdot \cos(\pi n x).
	\end{equation*}

	Після певних неприємних обчислень маємо:
	\begin{align*}
		a_n &= \phi_n = \frac{1}{\|X_n\|^2} \cdot \int_0^1 (x^2 - 1) \cdot \cos(\pi n x) \cdot \diff x = \\ &= \frac{1}{\frac{1}{2} + \frac{\sin(2 \pi n)}{4 \pi n}} \cdot \frac{2 \pi n \cdot \cos(\pi n) - 2 \sin(\pi n)}{\pi^3 \cdot n^3} = \\ &= 2 \cdot \frac{2 \pi n \cdot \cos(\pi n)}{\pi^3 \cdot n^3} = \frac{(-1)^n \cdot 4}{\pi^2 \cdot n^2}.
	\end{align*}

	Пригадаємо що у нас є ще $\lambda_0 = 0$ для якого $X_0 \equiv 1$, а також $T'(t) = 0$, тобто $T(t) = const = a_0$. Зрозуміло, що аналогічно попередньому ($a_n$, $n \in \NN$) маємо:
	\begin{equation*}
		a_0 = \phi_0 = \frac{1}{\|1\|^2} \cdot \int_0^1 (x^2 - 1) \cdot 1 \diff x = \left. \left( \frac{x^3}{3} - x \right) \right|_0^1 = - \frac{2}{3}.
	\end{equation*}

	Підставляємо це все у вигляд загального розв'язку:
	\begin{equation*}
		u(x, t) = -\frac{2}{3} - \frac{4}{\pi^2} \cdot \Sum_{n = 1}^\infty \frac{(-1)^n \cdot \cos(\pi n x) \cdot e^{ - (\pi n)^2 t}}{n^2}
	\end{equation*}
\end{solution}

\end{document}
