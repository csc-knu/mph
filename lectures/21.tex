\documentclass[a4paper, 12pt]{article}
\usepackage[utf8]{inputenc}
\usepackage[english, ukrainian]{babel}
\usepackage{amsmath, amssymb}
\usepackage[top = 2 cm, left = 1 cm, right = 1 cm, bottom = 2 cm]{geometry} 

\usepackage{fancyhdr}
\pagestyle{fancy}
\rhead{Нікіта Скибицький, ОМ-3}
\cfoot{\thepage}

\usepackage{multicol, graphicx}

\usepackage{amsthm}
\theoremstyle{definition}
\newtheorem{theorem}{Теорема}[subsection]
\newtheorem{definition}{Визначення}
\newtheorem{prove}{Доведення}
\newtheorem{problem}{\normalfont{\textit{Задача}}}[section]
\newtheorem*{solution}{Розв'язок}
\newtheorem{side_comment}{Зауваження}

\allowdisplaybreaks
\setlength\parindent{0pt}

\newcommand{\argmax}{\arg\max}
\newcommand{\argmin}{\arg\min}

\newcommand{\dif}{\mathrm{d}}
\newcommand{\dydx}{\dfrac{\dif y}{\dif x}}
\newcommand{\dxdt}{\dfrac{\dif x}{\dif t}}
\newcommand{\dydt}{\dfrac{\dif y}{\dif t}}

\newcommand{\NN}{\mathbb{N}} 
\newcommand{\ZZ}{\mathbb{Z}}
\newcommand{\QQ}{\mathbb{Q}}
\newcommand{\RR}{\mathbb{R}}
\newcommand{\CC}{\mathbb{C}}

\newcommand{\const}{\text{const}}

\newcommand{\LaReF}[1]{(\ref{#1})}

\renewcommand{\epsilon}{\varepsilon}
\renewcommand{\phi}{\varphi}

\newcommand{\ws}{\text{ }}

\newcommand{\Max}{\displaystyle\max\limits}
\newcommand{\Min}{\displaystyle\min\limits}
\newcommand{\Sum}{\displaystyle\sum\limits}
\newcommand{\Int}{\displaystyle\int\limits}
\newcommand{\Prod}{\displaystyle\prod\limits}

\newenvironment{system}{\begin{equation}\left\{\begin{aligned}}{\end{aligned}\right.\end{equation}}
\newenvironment{system*}{\begin{equation*}\left\{\begin{aligned}}{\end{aligned}\right.\end{equation*}}

\usepackage{color}

\usepackage{xfrac}

\title{{\Huge МАТЕМАТИЧНА ФІЗИКА}}
\author{Скибицький Нікіта}
\date{\today}

\usepackage{amsthm}
\usepackage[dvipsnames]{xcolor}
\usepackage{thmtools}
\usepackage[framemethod=TikZ]{mdframed}

\theoremstyle{definition}
\mdfdefinestyle{mdbluebox}{%
	roundcorner = 10pt,
	linewidth=1pt,
	skipabove=12pt,
	innerbottommargin=9pt,
	skipbelow=2pt,
	nobreak=true,
	linecolor=blue,
	backgroundcolor=TealBlue!5,
}
\declaretheoremstyle[
	headfont=\sffamily\bfseries\color{MidnightBlue},
	mdframed={style=mdbluebox},
	headpunct={\\[3pt]},
	postheadspace={0pt}
]{thmbluebox}

\mdfdefinestyle{mdredbox}{%
	linewidth=0.5pt,
	skipabove=12pt,
	frametitleaboveskip=5pt,
	frametitlebelowskip=0pt,
	skipbelow=2pt,
	frametitlefont=\bfseries,
	innertopmargin=4pt,
	innerbottommargin=8pt,
	nobreak=true,
	linecolor=RawSienna,
	backgroundcolor=Salmon!5,
}
\declaretheoremstyle[
	headfont=\bfseries\color{RawSienna},
	mdframed={style=mdredbox},
	headpunct={\\[3pt]},
	postheadspace={0pt},
]{thmredbox}

\declaretheorem[style=thmbluebox,name=Теорема,numberwithin=subsection]{theorem}
\declaretheorem[style=thmbluebox,name=Лема,sibling=theorem]{lemma}
\declaretheorem[style=thmbluebox,name=Твердження,sibling=theorem]{proposition}
\declaretheorem[style=thmbluebox,name=Закон,sibling=theorem]{law}
\declaretheorem[style=thmbluebox,name=Рівняння,sibling=theorem]{th_equation}
\declaretheorem[style=thmbluebox,name=Наслідок,sibling=theorem]{corollary}

\declaretheorem[style=thmredbox,name=Приклад,sibling=theorem]{example}
\declaretheorem[style=thmredbox,name=Приклади,sibling=example]{examples}

\declaretheorem[style=thmredbox,name=Властивість,sibling=theorem]{property}
\declaretheorem[style=thmredbox,name=Властивості,sibling=property]{properties}

\mdfdefinestyle{mdgreenbox}{%
	skipabove=8pt,
	linewidth=2pt,
	rightline=false,
	leftline=true,
	topline=false,
	bottomline=false,
	linecolor=ForestGreen,
	backgroundcolor=ForestGreen!5,
}
\declaretheoremstyle[
	headfont=\bfseries\sffamily\color{ForestGreen!70!black},
	bodyfont=\normalfont,
	spaceabove=2pt,
	spacebelow=1pt,
	mdframed={style=mdgreenbox},
	headpunct={ --- },
]{thmgreenbox}

\mdfdefinestyle{mdblackbox}{%
	skipabove=8pt,
	linewidth=3pt,
	rightline=false,
	leftline=true,
	topline=false,
	bottomline=false,
	linecolor=black,
	backgroundcolor=RedViolet!5!gray!5,
}
\declaretheoremstyle[
	headfont=\bfseries,
	bodyfont=\normalfont\small,
	spaceabove=0pt,
	spacebelow=0pt,
	mdframed={style=mdblackbox}
]{thmblackbox}

\declaretheorem[name=Вправа,sibling=theorem,style=thmblackbox]{exercise}
\declaretheorem[name=Запитання,sibling=theorem,style=thmgreenbox]{ques}
\declaretheorem[name=Зауваження,sibling=theorem,style=thmgreenbox]{remark}
\declaretheorem[name=Визначення,sibling=theorem,style=thmblackbox]{definition}

\newtheorem{problem}{Задача}[subsection]
\newtheorem{sproblem}[problem]{Задача}
\newtheorem{dproblem}[problem]{Задача}
\renewcommand{\thesproblem}{\theproblem$^{\star}$}
\renewcommand{\thedproblem}{\theproblem$^{\dagger}$}
\newcommand{\listhack}{$\empty$\vspace{-2em}} 

\theoremstyle{remark}
\newtheorem*{solution}{Розв'язок}


\begin{document}

\tableofcontents

\setcounter{section}{4}
\setcounter{subsection}{4}
\setcounter{subsubsection}{3}
% \setcounter{theorem}{30}
\setcounter{equation}{17}

\subsubsection{Функція Гріна задачі Діріхле для кулі}

Будемо розглядати граничну задачу
\begin{system}
	\Delta U(P) = 0, \quad |P| < R, \\
	\left. U(P) \right|_{|P| = R} = f(P).
\end{system}

Побудуємо функцію Гріна першої граничної задачі оператора Лапласа для кулі. \medskip

Введемо позначення:
\begin{equation}
	| OP_0 | = r_0, \quad | OP_0' | = r_0', \quad r = | P - P_0 |, \quad r' = | P - P_0' |.
\end{equation}
 
На довільному промені, який проходить через центр кулі точку $O$ розмістимо всередині кулі у точці $P_0$ одиничний точковий додатний заряд. Розглянемо точку $P_0'$ симетричну точці   відносно сфери.
\begin{figure}[H]
	\centering
	\includegraphics[width=.75\textwidth]{img/21-1.png}
\end{figure}

Це означає, що обидві точки лежать на одному промені, а їх відстані від центру сфери задовольняють співвідношенню
\begin{equation}
	r_0 \cdot r_0' = R^2.
\end{equation}

В $P_0'$ точці розмістимо від'ємний заряд величини $e$, яку оберемо виходячи з властивостей функції Гріна. \medskip

Запишемо потенціал електростатичного поля від суми зарядів:
\begin{equation}
	\Pi(P) = \frac{1}{4 \pi r} - \frac{e}{4 \pi r}.
\end{equation}

Обчислимо величину $e$ використовуючи теорему косинусів:
\begin{nalign}
	\Pi(P) &= \frac{1}{4 \pi} \left. \left( \frac{1}{\sqrt{\rho^2 + r_0^2 - 2 \rho r_0 \cos \gamma}} - \frac{e}{\sqrt{\rho^2 + \frac{R^4}{r_0^2} - 2 \rho \cdot \frac{R^2}{r_0} \cdot \cos \gamma}} \right) \right|_{\rho = R} = \\
	&= \frac{1}{4 \pi}  \left( \frac{1}{\sqrt{R^2 + r_0^2 - 2 R r_0 \cos \gamma}} - \frac{e}{\sqrt{R^2 + \frac{R^4}{r_0^2} - 2 R \cdot \frac{R^2}{r_0} \cdot \cos \gamma}} \right) = \\
	&= \frac{1}{4 \pi} \cdot \frac{1 - e \cdot \frac{r_0}{R}}{\sqrt{R^2 + r_0^2 - 2 R r_0 \cos \gamma}} = 0.
\end{nalign}

Остання рівність буде вірною, якщо $e = R / r_0$. \medskip

Таким чином функцію Гріна задачі Діріхле для кулі можна записати при знайденому значенні  величини зовнішнього заряду:
\begin{equation}
	G_1 (P, P_0) = \frac{1}{4\pi} \left( 1 / \sqrt{\rho^2 + r_0^2 - 2 \rho r_0 \cos \gamma} - 1 / \sqrt{R^2 + \frac{\rho^2 r_0^2}{R^2} - 2 \rho r_0 \cos \gamma} \right).
\end{equation}

Для знаходження формули інтегрального представлення обчислимо:
\begin{multline}
	\left. \frac{\partial G_1 (P, P_0)}{\partial n_P} \right|_{P \in S} = \left. \frac{\partial G_1 (P, P_0)}{\partial \rho} \right|_{\rho = R} = \\
	= \frac{1}{4 \pi} \left. \left( - \frac{\rho - r_0 \cos \gamma}{(\rho^2 + r_0^2 - 2 \rho r_0 \cos \gamma)^{3/2}} + \frac{\frac{\rho r_0^2}{R^2} - r_0 \cos \gamma}{\left(\frac{\rho^2 r_0^2}{R^2} + r_0^2 - 2 R r_0 \cos \gamma\right)^{3/2}} \right) \right|_{\rho = R} = \\
	= - \frac{1}{4 \pi R} \cdot \frac{R^2 - r_0^2}{(R^2 + r_0^2 - 2 R r_0 \cos \gamma)^{3/2}}.
\end{multline}

Для запису остаточної формули треба ввести сферичну систему координат. Запишемо через сферичні кути:
\begin{equation}
	\cos \gamma = \frac{\measuredangle (OP, OP_0)}{\rho r_0} = \cos \theta \cos \theta_0 + \sin \theta \sin \theta_0 \cos (\phi - \phi_0).
\end{equation}

Тут $\rho, \phi, \theta$ --- сферичні координати точки $P$, а $r_0, \phi_0, \theta_0$ --- сферичні координати точки $P_0$. \medskip

\begin{th_formula}[формула Пуассона для кулі]
	Використовуючи формулу (3.16) запишемо розв'язок задачі Діріхле:
	\begin{equation}
		U(r_0, \phi_0, \theta_0) = \frac{R}{4 \pi} \Int_0^{2 \pi} \Int_0^\pi \frac{(R^2 - r_0^2) \sin \theta f(\phi, \theta) \diff \theta \diff \phi}{(R^2 + r_0^2 - 2 R r_0 \cos \gamma)^{3/2}}.
	\end{equation}

	Ця формула дає розв'язок задачі Діріхле для рівняння Лапласа і називається \it{формулою Пуассона для кулі}.
\end{th_formula}

\subsubsection{Функція Гріна для областей на площині}

Функція Гріна для областей у двовимірному випадку в принципі можна будувати в той же спосіб, що і в тривимірному випадку. При цьому треба враховувати вигляд фундаментального розв'язку для $\RR^2$, що приводить до наступного вигляду функції Гріна:
\begin{equation}
	G_i (p, p_0) = \frac{1}{2 \pi} \cdot \ln \left( \frac{1}{|p - p_0|} \right) + g_i (p, p_0), \quad p, p_0 \in D \subset \RR^2.
\end{equation}

Фізичний зміст фундаментального розв'язку в двовимірному випадку представляє собою потенціал електростатичного поля в точці $p$ рівномірно зарядженої одиничним додатнім зарядом  нескінченої нитки, яка проходить ортогонально до площини через деяку точку $p_0$. Точки $p, p_0$ належать площині:
\begin{figure}[H]
	\centering
	\includegraphics[width=.75\textwidth]{img/21-2.png}
\end{figure}

Аналогічно кулі, можна отримати функцію Гріна задачі Діріхле для кола, яка має вигляд:
\begin{equation}
	G_1 (p, p_0) = \frac{1}{2 \pi} \left( \ln \left( \frac{1}{\sqrt{r_0^2 + \rho^2 - 2 \rho r_0 \cos \gamma}} \right) - \ln \left( \frac{R}{r_0} \cdot \frac{1}{\sqrt{\frac{R^4}{r_0^2} + \rho^2 - 2 \rho \cdot \frac{R^2}{r_0} \cdot \cos \gamma}} \right) \right)
\end{equation}

\end{document}